\documentclass[../thesis]{subfiles}
\graphicspath{{\subfix{../figures/}}}

\begin{document}
\chapter{Implementation}\label{ch:implementation}

\lettrine[lines=3]{\textcolor{Maroon}{R}}{elying} on the foundations of \SEE{} and the \glsentrylong{lsp} established in the previous chapter, we can now turn to the core part of this thesis:
The integration of \gls{lsp} into \SEE{}, with a special focus on how to build \glspl{city} using \gls{lsp}'s \glspl{capability}.
We will start by briefly going over some preliminary changes to both \SEE{} and the \gls{lsp} specification.
Then, we will spend a good chunk of this chapter specifying and explaining the algorithm which "converts" \gls{lsp} information into \glspl{city}, before looking into how additional \glspl{capability} can be integrated into \SEE{}'s \glspl{city} and \glspl{window} specifically.
Finally, we will conduct a brief technical evaluation, with a more thorough user study following in the next chapter.

\section{Preliminary Changes}
As promised in the preceding paragraph, we will first quickly list some preparations.

\subsection{Specification Cleanup}
While familiarizing myself with the \gls{lsp} specification, I noticed and fixed a few small issues along the way.
Most of these were of a formal nature (\eg, spelling, grammar, formatting, consistent usage of terms), some were fixing incorrect TypeScript syntax in the definition of \gls{lsp}'s data models.
The rest of the changes were related to the so-called snippet grammar.

In the context of \gls{lsp}, snippets are in essence string templates that are inserted on certain completions (see \cref{subsec:unplanned}), with some designed parts being filled in by the programmer on insertion.
There are also parts that can be filled in by certain values (\eg, the file name), which can themselves be transformed using regular expressions.\footnote{
	I am skipping over some additional features and details here because this is not that relevant a \gls{capability} for us---to get the full picture, see \web{https://microsoft.github.io/language-server-protocol/specifications/lsp/3.18/specification/\#snippet\_syntax}{2024-10-10}.
}
The complexity of the combinations of all these features increase the possibility of misunderstandings, which is why the snippet's grammar has been formally specified in \gls{ebnf}.
However, as it was written down in the specification, the grammar had a few problems that I have fixed.
Three notable examples are:
\begin{itemize}
	\item Some alternatives were incorrectly grouped, contradicting the explanatory text above them.
	      Also, the rules on how and when control characters had to be escaped were inconsistent with the surrounding text.\footnote{
		      This has lead to confusion in some projects making use of snippets.
		      See, for example, \web{https://github.com/neovim/neovim/issues/30495}{2024-10-10}.
	      }
	\item The grammar contained some string transformations that were unexplained in the text.
	      Since the \gls{lsp} specification is based on \gls{vscode}, I added explanations to the text based on what these transformations did in \gls{vscode}'s source code.
	\item Finally, there were ambiguities present in the grammar that led to \textsf{FIRST}/\textsf{FOLLOW} conflicts.
	      I have rewritten the grammar to eliminate these, and it should now be $LL(1)$-parseable~\cite[222--224]{aho2007}.

\end{itemize}


I have submitted these fixes as a pull request\footnote{\web{https://github.com/microsoft/language-server-protocol/pull/1886}{2024-10-10}}.
After addressing the resulting code review, it has been merged, and the changes will be incorporated in the upcoming 3.18 release of the specification.

\subsection{Preparing SEE}
There was not much I had to do in terms of getting \SEE{} ready, so this section will be very short:
\begin{itemize}
	\item I have integrated the OmniSharp \gls{lsp} C\# library\footnote{
		      Available at \web{https://github.com/OmniSharp/csharp-language-server-protocol}{2024-10-11}
	      } into \SEE{}, which we will leverage in the subsequent sections so that we can use \gls{lsp} without needing to worry about \gls{jrpc} encoding, data models, and so on.
	\item \Glspl{window} have previously been made editable by \textcite{moritz} in his bachelor thesis, also enabling collaborative editing over the internet.
	      I unfortunately had to remove these changes because they did not work anymore in the current version of \SEE{}, and additionally caused a lot of complexity overhead in the \gls{window} implementation that would have made the \gls{lsp} integration much harder to accomplish.
	\item Finally, the attribute space $\mathcal{A}$ in \SEE{} did not allow for \glspl{range} of the form \gls{lsp} needs, so I had to replace the existing attributes (which track the line and column, but not a full range) with a proper set of \gls{range} attributes.
	      In \cref{subsec:graph}, we have introduced this as a single \texttt{Source.Range} attribute, but in reality, there are four \gls{range} attributes---one per member of the decomposed form.
	      We will ignore this reality for the rest of this thesis and act like the \gls{range} is a single attribute, that is, for all project graphs with nodes $V$ and attributes $a$, it holds that $\{a(v, \texttt{Source.Range}) \mid v \in V \land (v, \texttt{Source.Range}) \in \mathrm{dom}(a) \} \subseteq \mathcal{R}$.
\end{itemize}

All of these changes have been made across two pull requests to \SEE{}.\footnote{
	\url{https://github.com/uni-bremen-agst/SEE/pull/687} and \web{https://github.com/uni-bremen-agst/SEE/pull/715}{2024-10-11}.
}

\section{Generating Code Cities using LSP}\label{sec:generate}
\fxfatal{}

\subsection{Algorithm}
\fxfatal{}
% Introduce and explain algorithm roughly first before linking/including it in detail.
%
% - LSP functions not by us
% - Domain specified for parameters (mathcal{S} for symbols)
%
% Explain special formatting of non-math parts:
% - SmallCaps: Procedures/functions
% - Gray italics: Comments
% - Bold: Keywords
% - Normal font: Strings
% - Typewriter: Properties of returned LSP JSON objects and attribute names (elements in A_K)
% Answer where we get documents for alg from
% Note what was omitted:
% - Details on adding nodes (e.g., IDs)
% - Configuration (e.g., excluding certain types)
% - Progress reporting
% - Simplification: Path, Range, and Name in SEE are actually Source.-prefixed

In \cref{alg:generate}.

\tikzexternaldisable

\begin{algorithm}
	\small

	\floatname{algorithm}{Algorithm}
	\caption{How \gls{city} graphs can be generated from \gls{lsp} information.}\label{alg:generate}
	\begin{algorithmic}[1]
		\Require  {Family of \textsc{Lsp} functions provided by the \gls{ls}}, {set of documents $D$}
		\Ensure {Graph $G$ representing the underlying software project}
		\Statex
		\State $V, E, a, s, t, \ell, C \gets \varnothing$ \Comment{Initialize empty graph components.}
		\label{alg:generate:begin}
		\ForAll{$d \in D$}
		\State \Call{LspOpenDocument}{$d$} \Comment{Document needs to be opened for all capabilities to work.}
		\State $v_d \gets$ \Call{AddDocumentNode}{$d$} \Comment{Each document becomes a node\dots}
		\ForAll{$x \in \textsc{LspDocumentSymbols}(d)$}
		\State \Call{MakeChild}{\Call{AddSymbolNode}{$x$}, $v_d$} \Comment{\dots with its symbols as children.}
		\EndFor
		\If{\Call{LspLanguageServerSupportsPullDiagnostics}{\null}}
		\State \Call{HandleDiagnostics}{\Call{LspPullDocumentDiagnostics}{$d$}}
		\Else
		\LComment{We will save any incoming diagnostics in the background and handle them at the end.}
		\State \Call{LspRegisterPushDiagnosticsCallback}{$d$, $(c) \mapsto (C \gets C \cup \{c\})$}
		\EndIf{}
		\State \Call{LspCloseDocument}{$d$}
		\EndFor
		\label{alg:generate:end1}

		\Statex
		\ForAll{$v \in V : (v, \tt{Source.Range}) \in \text{dom}(a)$}
		\LComment{First, connect nodes to each other based on LSP relations.}
		\State \Call{ConnectNodeVia}{\textsc{LspGoToDefinition}, {Definition}, $v$}
		\State \Call{ConnectNodeVia}{\textsc{LspGoToDeclaration}, {Declaration}, $v$}
		\State \Call{ConnectNodeVia}{\textsc{LspGoToTypeDefinition}, {TypeDefinition}, $v$}
		\State \Call{ConnectNodeVia}{\textsc{LspGoToImplementation}, {Implementation}, $v$}
		\State \Call{ConnectNodeVia}{\textsc{LspReferences}, {Reference}, $v$}

		\Statex
		\If{$a(v, \tt{Type}) = \text{Method}$}
		\Comment{We need to integrate the call hierarchy into the graph.}
		\State $I \gets \Call{LspPrepareCallHierarchy}{a(v, \tt{Source.Path}), a(v, \tt{Source.Range})}$
		\LComment{
			\textsc{GetMatchingItem} returns the item in $I$ with the same name and location as $v$.
		}
		\State $i \gets \Call{GetMatchingItem}{I, v}$
		\State $R \gets \Call{LspCallHierarchyOutgoingCalls}{i}$
		\State $V' \gets \bigcup\limits_{r \in R} \Call{FindNodesByLocation}{r\tt{.path}, r\tt{.range}}$
		\ForAll{$v' \in V'$}
		\State $\Call{AddEdge}{v, v', \text{Call}}$
		\EndFor
		\ElsIf{$a(v, \tt{Type}) = \text{Type}$}
		\Comment{We need to integrate the type hierarchy into the graph.}
		\State $I \gets \Call{LspPrepareTypeHierarchy}{a(v, \tt{Source.Path}), a(v, \tt{Source.Range})}$
		\State $i \gets \Call{GetMatchingItem}{I, v}$
		\State $R \gets \Call{LspTypeHierarchySupertypes}{i}$
		\State $V' \gets \bigcup\limits_{r \in R} \Call{FindNodesByLocation}{r\tt{.path}, r\tt{.range}}$
		\ForAll{$v' \in V'$}
		\State $\Call{AddEdge}{v, v', \text{Extend}}$
		\EndFor
		\EndIf{}
		\EndFor
		\label{alg:generate:end2}

		\Statex
		\State \Call{HandleDiagnostics}{$C$} \Comment{Handle diagnostics that were collected in the background.}
		\State \Call{AggregateMetrics}{\{\tt{Metric.LOC}\}}
		\State \Call{AggregateMetrics}{\{\tt{ErrorCount, WarningCount, InformationCount, HintCount}\}}
		\State \Return $(V, E, a, s, t, \ell)$
		\label{alg:generate:end3}

		\Statex
		\algstore{lspcity}
	\end{algorithmic}
\end{algorithm}

\begin{algorithm}
	\small
	\begin{algorithmic}[1]
		\algrestore{lspcity}
		\label{alg:generate:funcstart}
		\Function{AddDocumentNode}{$d \in D$}
		\State $v_d \gets$ \Call{NewNode}{\null}
		\State $a' \gets \varnothing$
		\State $a'(v_d, \tt{Type}) \gets$ File
		\State $a'(v_d, \tt{Source.Path}) \gets d$
		\State $a'(v_d, \tt{Metric.LOC}) \gets |\Call{ReadLines}{d}|$ \Comment{\textsc{ReadLines} returns the set of lines in the file.}
		\State $V \gets V \cup \{v_d\}$
		\State $a \gets a \cup a'$
		\State \Call{MakeChild}{$v_d$, \Call{AddDirectoryNode}{$d$\tt{.directory}}}
		\State \Return $v_d$
		\EndFunction

		\Statex
		\Function{AddSymbolNode}{$x \in \mathcal{S}$}
		\State $v \gets \Call{NewNode}{\null}$
		\State $a' \gets \varnothing$
		\State $a'(v, \tt{Source.Name}) \gets x\tt{.name}$
		\State $a'(v, \tt{Source.Path}) \gets d$
		\State $a'(v, \tt{Type}) \gets x\tt{.kind}$
		\State $a'(v, \tt{Deprecated}) \gets (\text{deprecated} \in x\tt{.tags})$
		\State $a'(v, \tt{Source.Range}) \gets x\tt{.range}$
		\State $a'(v, \tt{Metric.LOC}) \gets e_l^{x\tt{.range}} - b_l^{x\tt{.range}}$
		\LComment{Several other similar attributes omitted here...}
		\State $a'(v, \tt{HoverInfo}) \gets \Call{LspHover}{d, x\tt{.range}}$
		\If{$a' \nsubseteq a$} \Comment{If an isomorphic node does not already exist...}
		\State $V \gets V \cup \{v\}$
		\Comment{...add it and handle its children.}
		\State $a \gets a \cup a'$
		\ForAll{$x' \in x$\tt{.children}}
		\State \Call{MakeChild}{\Call{AddSymbolNode}{$x'$}, $v$}
		\Comment{Recurse.}
		\EndFor
		\EndIf
		\State \Return{$v$}
		\EndFunction

		\Statex
		\Function{MakeChild}{$v_c \in V, v_p \in V$}
		\LComment{The {partOf} edges must induce a tree structure.
			Hence, if a node already is a part of another node, we must not add another {partOf} edge.}
		\If{$\exists e \in E: \ell(e) = \text{partOf} \land s(e) = v_c$}
		\State \Output{Warning: Hierarchy is cyclic. Some children will be omitted.}
		\Else
		\State $\Call{AddEdge}{v_c, v_p, \text{partOf}}$
		\EndIf
		\EndFunction

		\Statex
		\Function{ConnectNodeVia}{$\textsc{LspFun} \in (D \times \mathcal{R})^{D \times \mathcal{R}}, l \in \Sigma, v \in V$}
		\LComment{Function $\textsc{LspFun}$ only returns locations, so we need to find the relevant nodes first.}
		\ForAll{$(d, r) \in \Call{LspFun}{a(v, \tt{Source.Path}), a(v, \tt{Source.Range})}$}
		\ForAll{$v' \in \Call{FindNodesByLocation}{d, r}$}
		\State $\Call{AddEdge}{v, v', l}$
		\EndFor
		\EndFor
		\EndFunction

		\Statex
		\Function{AddEdge}{$v_s \in V, v_t \in V, l \in \Sigma$}
		\State $e' \gets \Call{NewEdge}{\null}$
		\State $E \gets E \cup \{e'\}$
		\State $s(e') \gets v_s$
		\State $t(e') \gets v_t$
		\State $\ell(e') \gets l$
		\EndFunction
	\end{algorithmic}
\end{algorithm}

\begin{algorithm}
	\small
	\begin{algorithmic}[1]
		\Function{FindNodesByLocation}{$d \in D, r \in \mathcal{R}$}
		\LComment{We pick the nodes with the most specific range containing the given location.}
		\State $\text{getR}(v) = a(v, \tt{Source.Range})$
		\State $N \gets \{ v \in V \mid a(v, \tt{Source.Path}) = d \land b_l^r, e_l^r \in \left[b_l^{\text{getR}(v)}, e_l^{\text{getR}(v)}\right]$
		\Statex $\hphantom{N \gets \{ v \in V \mid a(v, \tt{Source.Path}) = d} \land\ b_c^r \geq b_c^{\text{getR}(v)} \land
			e_c^r \leq e_c^{\text{getR}(v)}
			\Big\}$
		\State {$N \gets \argmin\limits_{v \in N} e_l^{\text{getR}(v)} - b_l^{\text{getR}(v)}$}
		\State \Return{$\argmin\limits_{v \in N} e_c^{\text{getR}(v)} - b_c^{\text{getR}(v)}$}
		\EndFunction

		\Statex
		\Function{AddDirectoryNode}{$p \in \mathcal{A}_V$}
		\If{$\exists! v \in V: a(v, \tt{Source.Path}) = p$} \Comment{Check if node exists already.}
		\State \Return $v$ \Comment{If so, just pick that one.}
		\EndIf
		\State $v_p \gets$ \Call{NewNode}{\null}
		\State $a' \gets \varnothing$
		\State $a'(v_p, \tt{Type}) \gets$ Directory
		\State $a'(v_p, \tt{Source.Path}) \gets p$
		\State $V \gets V \cup \{v_p\}$
		\State $a \gets a \cup a'$
		\State $v_p^* \gets$ \Call{AddDirectoryNode}{\Call{GetParentDirectory}{$p$}}
		\Comment{Recurse to add parent directories.}
		\If{$v_p \neq v_p^*$}
		\State \Call{MakeChild}{$v_p$, $v_p^*$}
		\EndIf
		\State \Return $v_p$
		\EndFunction

		\Statex
		\Function{HandleDiagnostics}{$d \subset \mathcal{D}$}
		\ForAll{$c \in d$}
		\State $V_c \gets \Call{FindNodesByLocation}{c\tt{.path}, c\tt{.range}}$
		\ForAll{$v \in V_c$} \Comment{Save diagnostics count (grouped by severity) in all affected nodes.}
		\State $n \gets c\tt{.severity} + \text{Count}$  \Comment{Concatenate \emph{Count} to attribute name.}
		\If{$(v, n) \in \text{dom}(a)$}
		\State $a(v, n) \gets a(v, n) + 1$
		\Else
		\State $a(v, n) \gets 1$
		\EndIf
		\EndFor
		\EndFor
		\EndFunction

		\Statex
		\Function{AggregateMetrics}{$M \subset \mathcal{A}_K$}
		\ForAll{$v \in V: \nexists e \in E: t(e) = v \land \ell(e) = \text{partOf}$} \Comment{Aggregate from each root node.}
		\ForAll{$m \in M$}
		\State \Call{AggregateMetricFromRoot}{$m, v$}
		\EndFor
		\EndFor
		\EndFunction

		\Statex
		\Function{AggregateMetricFromRoot}{$m \in \mathcal{A}_K, v_r \in V$}
		\State $V_c \gets \{v \in V \mid \exists e \in E: t(e) = v_r \land s(e) = v \land \ell(e) = \text{partOf}\}$ \Comment{Immediate children.}
		\ForAll{$v \in V_c$}
		\State \Call{AggregateMetricFromRoot}{$m, v$}
		\EndFor
		\LComment{After the recursion above, immediate children now definitely have attribute $m$.}
		\If{$(v_r, m) \notin \text{dom}(a)$} \Comment{We don't want to overwrite existing metrics.}
		\State $a(v_r, m) \gets \sum\limits_{v \in V_c} a(v, m)$
		\EndIf
		\EndFunction
		\label{alg:generate:funcend}
	\end{algorithmic}
\end{algorithm}

\tikzexternalenable


\subsection{KD-Trees}
\fxfatal{}

\subsection{Code}
\fxfatal{}

\subsection{Editor UI}
\fxfatal{}

\subsection{Supported langs and so on}
\fxerror{Or maybe in tech eval instead}

\section{Integrating LSP Functionality into Code Cities}\label{sec:intocity}
\fxfatal{}

\section{Integrating LSP Functionality into Code Windows}\label{sec:intowindow}
\fxfatal{}

\section{Technical Evaluation}\label{sec:techeval}
\fxfatal{}

\section{Interim Conclusion}
\fxfatal{}
\fxerror{Also recap all written code/PRs here, maybe with LOC/links, maybe in a table.}

