%! TEX root = ../thesis.tex
\usepackage[toc,numberedsection=autolabel,section=chapter,abbreviations,xindy]{glossaries-extra}
%\usepackage{glossary-mcols}
%\usepackage{marginnote}
\usepackage{glossary-longextra}
\usepackage{marginfix}
\usepackage{float}

% This part is from https://tex.stackexchange.com/a/605352
\makeatletter
\newfloat{info@box}{tbp}{loi}[section]% 1: Name of float environment. 2: Default placement (top, bottom, ...). 3: File extension if written to an aux-file (like toc, lof, lot, loa, ...). 4: Numbering within <section/subsection/...>.
\makeatother
\floatname{info@box}{Infobox}% Adapt caption.

\newenvironment{infobox}[1]{% Create new environment using info@box and tcolorbox
	\begin{info@box}
		\begin{tcolorbox}[colback=Maroon!15!white,colframe=Maroon!75!black,
				arc=0mm, width=\textwidth, %left=1mm, right=1mm,
				boxrule=0.2mm, boxsep=0.3mm,
				toptitle=0.5mm, bottomtitle=0.2mm, lefttitle=1mm, righttitle=1mm,
				beforeafter skip=-2mm,
				fonttitle=\bfseries\sffamily\small,
				enhanced, drop fuzzy shadow,
				adjusted title=New Term: #1
			]%
			}{%
		\end{tcolorbox}
	\end{info@box}
}

\makeatletter
\newcommand*{\@starsymfn}[1]{%
	$\ifcase#1 \or*\or**\or\dagger\or\ddagger\or\mathsection\or\mathparagraph\or\|%
		\else\roman{footnoteG}\fi$%
}
\newcommand*{\starsymfn}[1]{%
	\expandafter\@starsymfn\csname c@#1\endcsname%
}
\makeatother


\DeclareNewFootnote{G}
\renewcommand*{\thefootnoteG}{\protect\firstuseformat{\protect\starsymfn{footnoteG}}}

% No links for description
\let\oldglsdesc\glsdesc
\renewcommand{\glsdesc}[1]{\oldglsdesc*{#1}}
\let\Oldglsdesc\Glsdesc
\renewcommand{\Glsdesc}[1]{\Oldglsdesc*{#1}}

% File index
\newglossary{file}{fln}{flo}{Attached Files}
\makeglossaries{}

% ----------------------------------------------------------------------------

% https://tex.stackexchange.com/a/359561

\let\fc\lowercase
\let\oldprintglossary\printglossary
\def\printglossary{\let\fc\uppercase\oldprintglossary}

% Full name on first time
%\setabbreviationstyle[acronym]{long-postshort-user}
\setabbreviationstyle[acronym]{long-short-desc}
% Also color acronym itself (or not)
% \renewcommand{\glsfirstabbrvuserfont}[1]{\textcolor{Maroon}{#1}}

\GlsXtrEnableInitialTagging{acronym}{\itag}

\newcommand{\glossterm}[1]{%
	\ifglshaslong{#1}%
	{\textit{\glsentrytext{#1}} (\glsxtrlong{#1})}%
	{\textit{\glsentrytext{#1}}}%
}

% Only color for the first time
\newcommand{\firstuseformat}[1]{\textcolor{Maroon}{{#1}}}
\newcommand{\seconduseformat}[1]{\textcolor{black}{{#1}}}
\newcommand{\mainformat}[1]{%
	\ifglsused{#1}{}{\glslinkvar{%
			\let\fc\uppercase%
			% \begin{infobox}{\glossterm{#1}}
			% 	\glsentrydesc{#1}
			% \end{infobox}%
			\footnoteG{\firstuseformat{\Glsentrytext{#1}}: \glsentrydesc{#1}}%
			\let\fc\lowercase}{}{}%
	}%
}


\defglsentryfmt[main]{\glsgenentryfmt\mainformat{\glslabel}}
%\defglsentryfmt[\acronymtype]{\glsgenentryfmt\ifglsused{\glslabel}{\glsentryshort{\glslabel}}{\glsentrylong{\glslabel}}\mainformat{\glslabel}}
\defglsentryfmt[\acronymtype]{\glsgenentryfmt\ifglsused{\glslabel}{}{\mainformat{\glslabel}}}

% Remove emphasis and set new color for files
\defglsentryfmt[file]{\emph{\textcolor{Cyan}{\glsentryname{\glslabel}}}}

\renewcommand{\glslinkpresetkeys}{%
	\ifglsused{\glslabel}%
	{\let\glstextformat\seconduseformat}%
	{\let\glstextformat\firstuseformat}%
}

% Special margin note for symbols
\newcommand{\glsmarginpar}[1]{
	\let\fc\uppercase%
	\marginpar{%
		\textcolor{Maroon}{\emph{\glsentryname{#1} (\glsentrysymbol{#1})}:} \glsentrydesc{#1}}%
	\let\fc\lowercase%
}


%%% ACRONYMS HERE %%%

\newacronym[description={\fc{A} three-item questionnaire asking participants for satisfaction with ease, completion time, and support information~\cite{lewis1991}.}]{asq}{ASQ}{After-Scenario Questionnaire}

\newacronym[description={\fc{A} protocol that can be viewed as the analogue to LSP for debuggers, with the goal to make it easier to integrate debuggers into development tools.}]{dap}{DAP}{Debug Adapter Protocol}

\newacronym[description={A syntax in which context-free grammars can be formally expressed.}]{ebnf}{EBNF}{Extended Backus–Naur Form}

\newacronym[description={The expected proportion of false positives within all positives.~\cite{benjamini1995}}]{fdr}{FDR}{False Discovery Rate}

\newacronym[description={The area of the world that can be seen by a camera or eye.}]{fov}{FOV}{Field of view}

\newacronym[description={The probability of getting at least one false positive in a family of statistical tests.~\cite{tukey1953}}]{fwer}{FWER}{Family-Wise Error Rate}

\newacronym[description={\fc{A} file format for graphs, used in \gls{see} for representing dependency and hierarchy graphs of software projects.}]{gxl}{GXL}{\itag{G}raph e\itag{X}change \itag{L}anguage}

\newacronym[description={\fc{E}ditor for source code with features that are useful for development (\eg, highlighting errors). Examples are VSCode or IntelliJ}.]{ide}{IDE}{Integrated Development Environment}

\newacronym[description={\fc{A} remote procedure call protocol that uses JSON as its encoding, supporting (among other features) asynchronous calls and notifications. It is used as the base for \gls{lsp} (even though \gls{lsp} is technically not a remote protocol).}]{jrpc}{JSON-RPC}{JavaScript Object Notation---Remote Procedure Call}

\newacronym[description={\fc{T}he number of lines in a source code file.}]{loc}{LOC}{Lines of Code}

\newacronym[description={\fc{A} format which language servers can emit to persist LSP-based information about a software project.}]{lsif}{LSIF}{Language Server Index Format}

\newacronym[description={\fc{A} protocol which specifies how language servers can provide language-specific features to IDEs, such as hover information, go to definition, or diagnostics.}]{lsp}{LSP}{Language Server Protocol}

\newacronym[description={\fc{T}he number of methods in a certain class.}]{nom}{NOM}{Number of Methods}

\newacronym[description={\fc{A} questionnaire that measures usability in sixteen questions across the three factors \emph{usefulness}, \emph{information quality}, and \emph{interface quality}.~\cite{lewis1992,lewis2002}}]{pssuq}{PSSUQ}{Post-Study System Usability Questionnaire}

\newacronym[description={\fc{A} \gls{poststudy} questionnaire measuring usability across 41 questions in its short version and 122 questions in its long version.~\cite{chin1988}}]{quis}{QUIS}{\itag{Q}uestionnaire for \itag{U}ser \itag{I}nteraction \itag{S}atisfaction}

\newacronym[description={\fc{A}n interactive software visualization tool using the code city metaphor in 3D, developed in the Unity game engine at the University of Bremen.}]{see}{SEE}{Software Engineering Experience}

\newacronym[description={A single question intended to estimate usability which uses a scale with 150 options.~\cite[53\psqq]{zijlstra1985a}}]{smeq}{SMEQ}{Subjective Mental Effort Question}

\newacronym[description={\fc{A} 50-item questionnaire measuring usability on the axes of efficiency, affect, help and support, steerability, and learnability.~\cite{kirakowski1994}}]{sumi}{SUMI}{Software Usability Measurement Inventory}

\newacronym[description={A simple questionnaire by \textcite{brooke1996} with ten Likert-scale questions that are supposed to measure the usability of a system.}]{sus}{SUS}{System Usability Scale}

\newacronym[description={A post-task questionnaire developed by NASA consisting of six questions asking about mental, physical, and temporal demand, as well as about performance, effort, and frustration level.~\cite[169]{hart1988}}]{tlx}{TLX}{\itag{T}ask \itag{L}oad Inde\itag{x}}

\newacronym[description={}]{ui}{UI}{User Interface}

\newacronym[description={A post-task questionnaire which does not depend on any pre-defined scale. Instead, numbers given by each user are put in relation to each other, with the resulting ratios serving as the usability measure.~\cite{mcgee2003}}]{ume}{UME}{Usability Magnitude Estimation}

\newacronym[description={A string that uniquely identifies some resource.}]{uri}{URI}{Uniform Resource Identifier}

\newacronym[description={A proprietary, but free \gls{ide} developed by Microsoft with a plugin system from which LSP originated. See \web{https://code.visualstudio.com}{2024-10-05}}]{vscode}{VSCode}{\itag{V}isual \itag{S}tudio \itag{Code}}


%%% GLOSSARY HERE %%%

\newglossaryentry{base}{name=base class,
	description={\fc{T}he base class of a class is its superordinate (\ie, above in the inheritance tree) class that itself has no further parent class within the project.},
	plural=base classes}

\newglossaryentry{city}{name=\fc{c}ode \fc{c}ity,
	description={\fc{I}n the code city metaphor, software components are visualized as buildings within a city, where various metrics of the software are represented visually (\eg, the height of a building could represent the lines of code of the component).},
	plural=code cities}

\newglossaryentry{ls}{name={L}anguage {S}erver,
description={\fc{A} locally-running JSON-RPC-based application following the Language Server Protocol that provides language-specific features and aids to the Language Client.}}

\newglossaryentry{lc}{name={L}anguage {C}lient,
description={\fc{A} development tool, such as an \gls{ide}, that supports \gls{lsp} and can hence integrate language-specific features into itself using compatible \glspl{ls}.}}

\newglossaryentry{capability}{name=\fc{c}apability,
	description={\fc{A} specific set of features a given \gls{ls} (and \gls{lc}) support.},
	plural=capabilities}

\newglossaryentry{window}{name=\fc{c}ode {w}indow, text=code window,
description={\fc{A} source code viewer in \SEE{} which supports very basic \gls{ide}-like functionality.}}

\newglossaryentry{fisherexact}{name={F}isher's exact test,
	description={\fc{A} test for statistical significance for contingency tables, intended as an alternative to the $\chi^2$ test to be accurate even with smaller sample sizes.~\cite{fisher1922}}}

\newglossaryentry{likert}{name={L}ikert scale,
	description={\fc{A} psychometric scale in which users indicate their agreement on a linear scale from "strongly agree" to "strongly disagree"~\cite{likert1932}.}}

\newglossaryentry{kdtree}{name={$k$-d tree},
	description={\fc{A} data structure (using a binary tree as a basis) with which certain spatial data in $k$ dimensions can be efficiently stored and retrieved.}}

\newglossaryentry{intervaltree}{name=interval tree,
	description={\fc{A} data structure meant to store intervals/ranges in such a way that overlapping or contained intervals can be found efficiently.}}

\newglossaryentry{multiplecomp}{name=multiple comparisons problem,
	description={\fc{A} problem in statistics that happens when the same dataset is tested for significant results multiple times, thereby increasing the chance of a false positive beyond what is acceptable.~\cite{tukey1953}}}

\newglossaryentry{reflexion}{name=\fc{r}eflexion \fc{a}nalysis,
	description={\fc{T}he process of comparing the architecture and implementation of a software project and finding incongruencies between the two.}}

\newglossaryentry{polytree}{name=\fc{p}olytree,
	description={\fc{A} directed acylic graph which also has no undirected cycles.}}

\newglossaryentry{editor}{name={U}nity Editor,
	description={\fc{T}he main \gls{ui} of the Unity game engine, in which scenes can be set up, components can be configured, the game itself can be run, etc. Note that it is only used for development purposes, and hence not included within generated builds of a game.}}

\newglossaryentry{hawthorne}{name=Hawthorne effect,
	description={\fc{T}he effect through which participants in a study behave differently when under observation by a researcher.
			The name is based on experiments conducted at the Hawthorne Plant in the 1930's~\cite{hawthorne1939}, although the effects seen there have later turned out to be likely unrelated to the observation~\cite{jones1992}.
		}}

\newglossaryentry{poststudy}{name=post-study,
	description={\fc{A} questionnaire for participants which is answered at the end of the study, after every task has been completed.}}

\newglossaryentry{posttask}{name=post-task,
	description={\fc{A} questionnaire for participants which is answered after each task.}}

\newglossaryentry{provider}{name=\fc{G}raph provider,
	description={\fc{A} component in \SEE{} that produces or transforms a project graph by some means. Can optionally take configuration parameters by users.}}

\newglossaryentry{singleton}{name={s}ingleton,
	description={\fc{a} class for which only one instance exists for the duration of the process.}}

\newglossaryentry{smell}{name=\fc{c}ode \fc{s}mell,
	description={\fc{c}ertain structures in source code that suggest that a refactoring is in order, such as duplicated code, or a very long method~\cite[85-87]{fowler2019}.}}

\newglossaryentry{mwu}{name={Mann-Whitney $U$ test},
	description={A test to check whether two distributions have identical distributions, requiring only ordinal-scaled independent samples drawn from the distributions.~\cite{mann1947}}}

\newglossaryentry{range}{name=Source Range, text=range,
	description={\fc{t}he contiguous portion within a source code file that a certain element occupies. In this thesis, we will express it as an ordered pair of a start and end position $(s, e)$, or, alternatively, as a 4-tuple $(s_l, s_c, e_l, e_c)$.
			In the latter representation, the first two elements describe the zero-indexed start line and start character offset, respectively, and the last two describe the corresponding (exclusive) end line and end character offset.}}

\newglossaryentry{token}{name=\fc{s}emantic token, text=semantic token,
	description={An \gls{lsp} \gls{capability} that returns tokens for a document, with the intention that those tokens can be used to apply syntax highlighting to the file. Apart from token types, it also offers token \emph{modifiers}, which can be used to apply additional formatting on top of the colors from the types.}}

\newglossaryentry{violin}{name=violin plot,
	description={A plot visualizing the distribution of a collection of data points along with an estimated probability density~\cite{hintze1998}.
			The black/white data points are randomly "jittered" along the $x$-axis to make them more differentiable from one another.
			A bigger, green point marks the average of the dataset.
		}}

\newglossaryentry{xlsform}{name=XLSForm,
	description={\fc{A} form standard based on human-readable Excel spreadsheets, which allows for complex forms (\eg, conditionals, skip logic) to be built~\cite{marder2024}.
			The forms can then be converted into the open ODK XForm format, which is compatible with a number of data collection tools~\cite{odkcommunity2019}.}}

%%% DIGITAL HERE %%%
\newglossaryentry{benchmark}{name={\tt{benchmark.zip}}, type=file,
	description={A ZIP archive containing both the script and all measured data of the technical evaluation. Refer to \cref{app:techdata} for more information.}}
\newglossaryentry{extract}{name={\tt{extract-metrics.py}}, type=file,
	description={A Python script which extracts metrics from a \gls{gxl} file and outputs them into a CSV file.}}
\newglossaryentry{psi}{name={\tt{SEE-VSCode\_redacted.csv}}, type=file,
	description={The pseudonymized participation data for group $\Psi$, collected from KoboToolbox during the user study.}}
\newglossaryentry{omega}{name={\tt{VSCode-SEE\_redacted.csv}}, type=file,
	description={The pseudonymized participation data for group $\Omega$, collected from KoboToolbox during the user study.}}
\newglossaryentry{psitest}{name={\tt{Eval-See-Vscode.xlsx}}, type=file,
	description={The \gls{xlsform} questionnaire for the $\Psi$ version of the user study.}}
\newglossaryentry{omegatest}{name={\tt{Eval-Vscode-See.xlsx}}, type=file,
	description={The \gls{xlsform} questionnaire for the $\Omega$ version of the user study.}}
\newglossaryentry{analysis}{name={\tt{analysis.zip}}, type=file,
	description={A ZIP archive containing all scripts for the analysis of the user study. Refer to \cref{app:scripts} for more information.}}
\newglossaryentry{redirect}{name={\tt{redirect-evaluation.py}}, type=file,
	description={A Python script which acts as a web server that redirects users to either the $\Psi$ or $\Omega$ version of the study, as described by \cref{alg:redirect}.}}
