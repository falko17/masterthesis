%! TEX root = ../thesis.tex
\usepackage[toc,numberedsection=autolabel,section=chapter,abbreviations]{glossaries-extra}
%\usepackage{glossary-mcols}
%\usepackage{marginnote}
\usepackage{glossary-longextra}
\usepackage{marginfix}
\usepackage{float,tcolorbox}

\tcbuselibrary{skins}

% This part is from https://tex.stackexchange.com/a/605352
\makeatletter
\newfloat{info@box}{tbp}{loi}[section]% 1: Name of float environment. 2: Default placement (top, bottom, ...). 3: File extension if written to an aux-file (like toc, lof, lot, loa, ...). 4: Numbering within <section/subsection/...>.
\makeatother
\floatname{info@box}{Infobox}% Adapt caption.

\newenvironment{infobox}[1]{% Create new environment using info@box and tcolorbox
	\begin{info@box}
		\begin{tcolorbox}[colback=Maroon!15!white,colframe=Maroon!75!black,
				arc=0mm, left=1mm, width=\textwidth,
				before skip balanced=0mm, beforeafter skip=-2mm,
				fonttitle=\bfseries\sffamily,
				enhanced, drop fuzzy shadow,
				title=New Term: #1
			]%
			}{%
		\end{tcolorbox}
	\end{info@box}
}

% No links for description
\let\oldglsdesc\glsdesc
\renewcommand{\glsdesc}[1]{\oldglsdesc*{#1}}
\let\Oldglsdesc\Glsdesc
\renewcommand{\Glsdesc}[1]{\Oldglsdesc*{#1}}

% File index
\newglossary{file}{fln}{flo}{Attached Files}
\makeglossaries{}

% ----------------------------------------------------------------------------

% https://tex.stackexchange.com/a/359561

\let\fc\lowercase
\let\oldprintglossary\printglossary
\def\printglossary{\let\fc\uppercase\oldprintglossary}

% Full name on first time
\setabbreviationstyle[acronym]{long-postshort-user}
% Also color acronym itself
\renewcommand{\glsfirstabbrvuserfont}[1]{\textcolor{Maroon}{#1}}

\newcommand{\glossterm}[1]{%
	\ifglshaslong{#1}%
	{\textit{\glsentrytext{#1}} (\glsxtrlong{#1})}%
	{\textit{\glsentrytext{#1}}}%
}

% Only color for the first time
\newcommand{\firstuseformat}[1]{\textcolor{Maroon}{{\emph{#1}}}}
\newcommand{\seconduseformat}[1]{\textcolor{black}{{#1}}}
\newcommand{\mainformat}[1]{%
	\ifglsused{#1}{}{\glslinkvar{%
			\let\fc\uppercase%
			\begin{infobox}{\glossterm{#1}}
				% TODO: Full title for acronyms
				\glsentrydesc{#1}
			\end{infobox}%
			\let\fc\lowercase}{}{}%
	}%
}


\defglsentryfmt[main]{\glsgenentryfmt\mainformat{\glslabel}}
%\defglsentryfmt[\acronymtype]{\glsgenentryfmt\ifglsused{\glslabel}{\glsentryshort{\glslabel}}{\glsentrylong{\glslabel}}\mainformat{\glslabel}}
\defglsentryfmt[\acronymtype]{\glsgenentryfmt\ifglsused{\glslabel}{}{\mainformat{\glslabel}}}

% Remove emphasis and set new color for files
\defglsentryfmt[file]{\emph{\textcolor{Cyan}{\glsentryname{\glslabel}}}}

\renewcommand{\glslinkpresetkeys}{%
	\ifglsused{\glslabel}%
	{\let\glstextformat\seconduseformat}%
	{\let\glstextformat\firstuseformat}%
}

% Bei Symbol spezielle Randnotiz
\newcommand{\glsmarginpar}[1]{
	\let\fc\uppercase%
	\marginpar{%
		\textcolor{Maroon}{\emph{\glsentryname{#1} (\glsentrysymbol{#1})}:} \glsentrydesc{#1}}%
	\let\fc\lowercase%
}


%%% ACRONYMS HERE %%%

\newacronym[description={\fc{A}n interactive software visualization tool using the code city metaphor in 3D, developed in the Unity game engine at the University of Bremen.}]{see}{SEE}{Software Engineering Experience}

\newacronym[description={\fc{T}he number of lines in a source code file.}]{loc}{LOC}{Lines of Code}

\newacronym[description={\fc{A} file format for graphs, used in \gls{see} for representing dependency and hierarchy graphs of software projects.}]{gxl}{GXL}{Graph eXchange Language}

\newacronym[description={\fc{E}ditor for source code with features that are useful for development (\eg, highlighting errors). Examples are \emph{Eclipse} or \emph{\textsc{JetBrains} IntelliJ}}.]{ide}{IDE}{Integrated Development Environment}

\newacronym[description={\fc{A} protocol which specifies how language servers can provide language-specific features to IDEs, such as hover information, go to definition, or diagnostics.}]{lsp}{LSP}{Language Server Protocol}

\newacronym[description={\fc{A} remote procedure call protocol that uses JSON as its encoding, supporting (among other features) asynchronous calls and notifications. It is used as the base for \gls{lsp} (even though \gls{lsp} is technically not a remote protocol).}]{jrpc}{JSON-RPC}{JavaScript Object Notation---Remote Procedure Call}

\newacronym{ui}{UI}{User Interface}


%%% GLOSSARY HERE %%%

\newglossaryentry{city}{name=\fc{c}ode \fc{c}ity,
	description={\fc{I}n the code city metaphor, software components are visualized as buildings within a city, where various metrics of the software are represented visually (\eg, the height of a building could represent the lines of code of the component).},
	plural=code cities}

\newglossaryentry{ls}{name={L}anguage {S}erver,
description={\fc{A} locally-running JSON-RPC-based application following the Language Server Protocol that provides language-specific features and aids to the Language Client.}}

\newglossaryentry{lc}{name={L}anguage {C}lient,
description={\fc{A} development tool, such as an \gls{ide}, that supports \gls{lsp} and can hence integrate language-specific features into itself using compatible \glspl{ls}.}}

\newglossaryentry{capability}{name=\fc{c}apability,
	description={\fc{A} specific set of features a given \gls{ls} (and \gls{lc}) support.},
	plural=capabilities}

\newglossaryentry{window}{name=\fc{c}ode \fc{w}indow,
	description={\fc{A} source code viewer in \SEE{} which supports very basic \gls{IDE}-like functionality.}}

\newglossaryentry{reflexion}{name=\fc{r}eflexion \fc{a}nalysis,
	description={\fc{T}he process of comparing the architecture and implementation of a software project and finding incongruencies between the two.}}


% TODO Replace with English source, change quote
\newglossaryentry{smell}{name=\fc{c}ode \fc{s}mell,
	description={\fc{c}ertain structures in source code that make it apparent that a refactoring is in order~\cite{Refa2020}.}}

%%% DIGITAL HERE %%%
\newglossaryentry{seezip}{name={\texttt{SEE.zip}}, type=file,
	description={\fc{T}he build of \SEE{} used for the evaluation.}}
