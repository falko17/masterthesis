\documentclass[../thesis]{subfiles}
\graphicspath{{\subfix{../figures/}}}

\begin{document}
\chapter{User Study}\label{ch:evaluation}

\lettrine[lines=3]{\textcolor{Maroon}{D}}{espite} having already done a technical evaluation in \cref{sec:techeval}, it is usually also a good idea to compare new approaches with existing state-of-the-art tools in a user study.
In our case, we want to compare \SEE{}'s \gls{lsp}-generated \glspl{city} with the \glspl{capability} a normal \gls{lsp}-enabled \gls{ide} offers.
For the latter, the Microsoft-developed \gls{ide} \gls{vscode} is a good fit, given that the \glsentrylong{lsp} itself originates here (see \cref{sec:lsp}) and that it still has a deep integration with it.

First, we will outline the general aim of this study by going over some existing related research, explaining important aspects of \gls{vscode}, and then enumerating our hypotheses.
Next, we will explain the details of the design of the study itself, before analyzing its results with the \participants{} participants in detail.
Finally, we describe some relevant threats to validity.

\section{Plan}  % TODO: Different title?
Our main aim here is to answer our second research question that we defined in \cref{sec:goals}:
\begin{displayquote}
	Are \glspl{city} a suitable means to present \gls{lsp} information to developers as compared to \glspl{ide} + tables (on the dimensions of speed, accuracy, and usability)?
\end{displayquote}

To empirically evaluate this research question, we will devise a series of short software engineering related tasks.
Participants then get randomly assigned to either use \SEE{} (along with our implementation from \cref{ch:implementation}) or \gls{vscode} (with an active \gls{ls}).
However, evaluating the supported \glspl{capability} (see \cref{tab:capabilities}) in this way turns out to be quite difficult---for example, how would one evaluate the \emph{Hover} \gls{capability}, let alone features like \glspl{token} which are almost identically implemented across \SEE{} and \gls{vscode}?
For this reason, we will abstain from incorporating the \gls{window}-related \glspl{capability} from \cref{sec:intowindow}.
Limiting ourselves, then, to the \gls{city}-related changes from \cref{sec:intocity}, we have:
\begin{enumerate}
	\item \emph{Diagnostics} being displayed as erosion icons above corresponding nodes.
	      \begin{itemize}
		      \item This feature was essentially already evaluated in my bachelor's thesis, albeit with the Axivion Dashboard as a data source instead of \gls{lsp}~\cite{galperin2021,galperin2022}.
	      \end{itemize}
	\item \emph{Hover} details being displayed when the user hovers the mouse above a node.
	      \begin{itemize}
		      \item Since this is used here almost identically as in \glspl{window}, it does not make much sense to compare it against \gls{vscode}.
	      \end{itemize}
	\item \emph{Go to location}, \emph{references}, and \emph{call/type hierarchy} being used for rendered edges and context menu actions.
	      \begin{itemize}
		      \item The context menu actions are not interesting to evaluate for the same reasons as above, though this does not apply to the generated edges.
	      \end{itemize}
\end{enumerate}

It appears that the only \gls{capability} reasonably evaluable in a user study of this form are actually the ones used in the generation of the \gls{city} in \cref{sec:generate}.
Besides, the bulk of the implementation pertains to the generation of \glspl{city}, so it makes sense to focus on them here.
As a result, the user study is now actually of a form (directly comparing \glspl{city} with \glspl{ide}) that has been researched in previous literature before, so let us take a look at that research first before designing our own study.

\subsection{Existing Research}
\fxfatal{}

\subsection{VSCode}
\fxfatal{}

\subsection{Hypotheses}
\fxfatal{}

\section{Structure}
\fxfatal{}

\subsection{Questionnaire}
\fxfatal{}

\subsection{Tasks}
\fxfatal{}

\section{Results}
\fxfatal{}

\subsection{Demographics}
\fxfatal{}

\pgfplotsset{every axis/.append style={
			%axis line style={-stealth},
			%label style={font=\itseries},
			tick label style={font=\footnotesize,text=gray!50!black},
		}}

\violinab{age}{Age}{20}{40}

\violinab{total-time}{Total Participation Time (hours)}{0}{5}

% TODO: ordinal data

\violinsv{a1-time}{Time (Minutes)}{0}{20}
\violinsv{a2-time}{Time (Minutes)}{0}{20}
\violinsv{a3-time}{Time (Minutes)}{0}{10}
\violinsv{a4-time}{Time (Minutes)}{0}{10}
\violinsv{a5-time}{Time (Minutes)}{0}{20}
\violinsv{a6-time}{Time (Minutes)}{0}{10}

\violinsus{}

\subsection{Correctness}
\fxfatal{}

\subsection{Time}
\fxfatal{}

\subsection{Usability}
\fxfatal{}

\fxfatal{Also a section on the effect of experience (and others)}

\subsection{Comments}
\fxfatal{}

\section{Threats to Validity}
\fxfatal{}

\section{Interim Conclusion}  % Or maybe just recap?
\fxfatal{}
