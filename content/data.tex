\chapter{Additional Information}

\lettrine[lines=3]{\textcolor{Maroon}{T}}{his} appendix collects a variety of additional information that I did not want to put within the main text of the thesis, in the fear that it breaks up the reading flow too much.
Hence, there are various disparate sections in here that bear no direct relation to one another.

\section{All \gls{sus} Questions}\label{app:susq}
Here are all ten questions of the \glsentrylong{sus}:

\begin{enumerate}
	\item "I think that I would like to use the system frequently."
	\item "I found the system unnecessarily complex."
	\item "I thought the system was easy to use."
	\item "I think that I would need the support of a technical person to be able to use the system."
	\item "I found the various functions in the system were well integrated."
	\item "I thought there was too much inconsistency in the system."
	\item "I would imagine that most people would learn to use the system very quickly."
	\item "I found the system very cumbersome to use."
	\item "I felt very confident using the system."
	\item "I needed to learn a lot of things before I could get going with the system."
\end{enumerate}

\section{Generation Time Without Edges}\label{app:noedge}
The breakdown of the generation algorithm components for the sample projects in the technical evaluation in \cref{sec:techeval} made it hard to tell the distribution of other components, since the edge components took up so much space.
For this reason, we present another version of the diagram in \cref{fig:techevalnoedges} with the edge component removed.

\begin{figure}
	\begin{subfigure}[T]{0.5\textwidth}
		\begin{center}
			\begin{tikzpicture}
				\begin{axis}[
						ylabel={Time in seconds},
						height=10cm,
						ybar stacked,
						name=bars,
						set layers,
						ymin=0, ymax=25,
						axis x line*=bottom,
						axis y line*=left,
						enlarge x limits={0.5},
						ymajorgrids,
						bar width=0.6cm,
						x tick label style={rotate=45, anchor=east, align=left, font=\scriptsize, yshift=-2},
						xtick=data,
						width=\textwidth,
						xticklabels={\proptt{aaoffline}-$O$, \proptt{aaoffline}-$B$, \proptt{dcaf-rs}-$O$, \proptt{dcaf-rs}-$B$},
						xtick={0,1,3,4},
					]

					\addplot+[fill] table [x=index,y=LSP Nodes,col sep=tab] {benchmark/rust.dat};
					\addplot+[fill] table [x=index,y=LSP Diagnostics,col sep=tab] {benchmark/rust.dat};
					\addplot+[fill] table [x=index,y=LSP Aggregate,col sep=tab] {benchmark/rust.dat};
					\addplot+[fill] table [x=index,y=LSP Tree,col sep=tab] {benchmark/rust.dat};
					\addplot+[fill] table [x=index,y=LSP Miscellaneous,col sep=tab] {benchmark/rust.dat};
				\end{axis}
				\plotornaments{bars}
			\end{tikzpicture}
		\end{center}
	\end{subfigure}
	\begin{subfigure}[T]{0.5\textwidth}
		\begin{center}
			\begin{tikzpicture}
				\begin{axis}[
						height=10cm,
						ybar stacked,
						name=bars,
						set layers,
						ymin=0, ymax=10,
						axis x line*=bottom,
						axis y line*=left,
						enlarge x limits={0.5},
						ymajorgrids,
						bar width=0.6cm,
						x tick label style={rotate=45, anchor=east, align=left, font=\scriptsize, yshift=-2},
						xtick=data,
						width=\textwidth,
						xticklabels={Bachelor-$O$, Bachelor-$B$, Master-$O$, Master-$B$},
						xtick={0,1,3,4},
						legend entries={Nodes, Diagnostics, Aggregation, Tree Creation, Miscellaneous},
						legend style={nodes={scale=0.7, transform shape}, at={(0.95,0.9)}}
					]

					\addplot+[fill] table [x=index,y=LSP Nodes,col sep=tab] {benchmark/tex.dat};
					\addplot+[fill] table [x=index,y=LSP Diagnostics,col sep=tab] {benchmark/tex.dat};
					\addplot+[fill] table [x=index,y=LSP Aggregate,col sep=tab] {benchmark/tex.dat};
					\addplot+[fill] table [x=index,y=LSP Tree,col sep=tab] {benchmark/tex.dat};
					\addplot+[fill] table [x=index,y=LSP Miscellaneous,col sep=tab] {benchmark/tex.dat};
				\end{axis}
				\plotornaments{bars}
			\end{tikzpicture}
		\end{center}
	\end{subfigure}\\
	\begin{subfigure}[T]{0.5\textwidth}
		\begin{center}
			\begin{tikzpicture}
				\begin{axis}[
						ylabel={Time in seconds},
						height=10cm,
						ybar stacked,
						name=bars,
						set layers,
						ymin=0, ymax=200,
						xmin=0, xmax=1,
						axis x line*=bottom,
						axis y line*=left,
						enlarge x limits={1},
						ymajorgrids,
						bar width=0.6cm,
						x tick label style={rotate=45, anchor=east, align=left, font=\scriptsize, yshift=-2},
						xtick=data,
						width=\textwidth,
						xticklabels={JabRef-$O$, JabRef-$B$},
						xtick={0,1},
					]

					\addplot+[fill] table [x=index,y=LSP Nodes,col sep=tab] {benchmark/java.dat};
					\addplot+[fill] table [x=index,y=LSP Diagnostics,col sep=tab] {benchmark/java.dat};
					\addplot+[fill] table [x=index,y=LSP Aggregate,col sep=tab] {benchmark/java.dat};
					\addplot+[fill] table [x=index,y=LSP Tree,col sep=tab] {benchmark/java.dat};
					\addplot+[fill] table [x=index,y=LSP Miscellaneous,col sep=tab] {benchmark/java.dat};
				\end{axis}
				\plotornaments{bars}
			\end{tikzpicture}
		\end{center}
	\end{subfigure}
	\begin{subfigure}[T]{0.5\textwidth}
		\begin{center}
			\begin{tikzpicture}
				\begin{axis}[
						height=10cm,
						ybar stacked,
						name=bars,
						set layers,
						ymin=0, ymax=250,
						axis x line*=bottom,
						axis y line*=left,
						xmin=3, xmax=4,
						enlarge x limits={1},
						ymajorgrids,
						bar width=0.6cm,
						x tick label style={rotate=45, anchor=east, align=left, font=\scriptsize, yshift=-2},
						xtick=data,
						width=\textwidth,
						xticklabels={SpotBugs-$O$, SpotBugs-$B$},
						xtick={3,4},
					]

					\addplot+[fill] table [x=index,y=LSP Nodes,col sep=tab] {benchmark/java.dat};
					\addplot+[fill] table [x=index,y=LSP Diagnostics,col sep=tab] {benchmark/java.dat};
					\addplot+[fill] table [x=index,y=LSP Aggregate,col sep=tab] {benchmark/java.dat};
					\addplot+[fill] table [x=index,y=LSP Tree,col sep=tab] {benchmark/java.dat};
					\addplot+[fill] table [x=index,y=LSP Miscellaneous,col sep=tab] {benchmark/java.dat};
				\end{axis}
				\plotornaments{bars}
			\end{tikzpicture}
		\end{center}
	\end{subfigure}
	\caption{Generation time for each project, broken down by parts of the algorithm, excluding the edge generation component.
		The suffix \emph{O} denotes the optimized (\gls{intervaltree}) version of the algorithm, while \emph{B} refers to the brute-force version.
	}\label{fig:techevalnoedges}
\end{figure}
